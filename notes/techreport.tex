\title{DRAFT: Not read yet: Scala-Gopher: make synchronous code work asynchnoiously with CSP }
\author{
        Ruslan Shevchenko \\
                ruslan@shevchenko.kiev.ua\\
        Kiev, Ukraine
}
\date{\today}



\documentclass[12pt]{article}
\usepackage{listings}
\usepackage{amsmath}
\usepackage{amssymb}
\usepackage{pgf}
\usepackage{tikz}
\usepackage{fancyvrb}
\usetikzlibrary{arrows,automata}
\usepackage[backend=bibtex]{biblatex}
\addbibresource{techreport.bib}

% "define" Scala
\lstdefinelanguage{scala}{
  morekeywords={abstract,case,catch,class,def,%
    do,else,extends,false,final,finally,%
    for,if,implicit,import,match,mixin,%
    new,null,object,override,package,%
    private,protected,requires,return,sealed,%
    super,this,throw,trait,true,try,%
    type,val,var,while,with,yield},
  otherkeywords={=>,<-,<\%,<:,>:,\#,@},
  sensitive=true,
  morecomment=[l]{//},
  morecomment=[n]{/*}{*/},
  morestring=[b]",
  morestring=[b]',
  morestring=[b]"""
}

\newcommand{\To}{\Rightarrow}


\begin{document}
\maketitle


\section{Introduction}

 Scala-gopher is a library-level implementation of process algebra [Communication Sequential Processes, see \cite{Hoare85communicatingsequential} as ususally enriched by $\pi$-calculus \cite{Milner:1992:CMP:162037.162038} naming primitives] in scala. In addition to support of a 'limbo/go-like' \cite{Inferno:Limbo}  \cite{golang} channels/goroutine programming style scala-gopher provide set of operations following typical scala idiomatic. 

    At first, lets remind the basic of 'go-like' CSP parallelism. The main entities of this model are channels, coroutines and selectors. Coroutines are lightweight threads of execution, which can communicate with each other by passing messages between channels. Channels can be viewed as blocked multiproducer/multiconsumer queues. Sending message to unbuffered channel suspend producing coroutines until moment, when this message will has been readed by some consumer. Buffered channels transfer control flow between sinks not on each message, but when internal channel buffer is full.  In such way communication via channel implicitly provide flow control functionality.  At last, selector is a way of coordination of several communication activities: like unix select(2) system call, select statement suspend current coroutines until one of actions (reading/writing to one of channel) will be possible.

   Let's look at the one simple example:
\begin{Verbatim}[fontsize=\small]
 def nPrimes(n:Int):Future[List[Int]]=
 {
    val in = makeChannel[Int]()
    val out = makeChannel[Int]()
    go {
      for(i <- 1 to n*n) out.write(i)
    }
    go {
      select.fold(in){ (ch,s) =>
        s match {
          case p:ch.read => out.write(p)
                            ch.filter(_ % p != 0)
        }
      }
    }
    go {
      for(i <- 1 to n) yield out.read
    }
 }
\end{Verbatim}
  Here two channels anf three gorautines are created.  The first coroutine just generate sequential numbers and send one to channel \verb|in|, second - keep channel in fold state and for each number which has been readed from state channel, write one to \verb|out| and produce next step by applying yet one filter to previous. And third just map range to values to receive list of first \verb|n| primes in Future.

  If we will look on sequence of steps during code evaluation, we will see at first generation of
number, then sequence of checks and then if number was prime - final output.  Note, that goroutine is different from jvm thread of execution: sequential code chunks are executed in configurable executor service; switching between chunks does not use blocking operations.

\section{Implementation of base constructs }


\subsection{Go: Translations of hight-order functions to asynchonious form.}

 The main entity of CSP is a 'process' which can be viewed as a block of code which handle specific events.  In Go CSP processes are represented as goroutines (aka coroutines). 

 \verb|go[X]{x:X}:Future[X]| is a think wrapper arround  SIP-22 async/await which do some preprocessing before async transformation:
\begin{itemize}
 \item do transformation of hight-order function in async form.


  Let $f(A \To B)\To C$ is a hight-order function, which accepts other function 
   $g:A \To B$ as parameter. 
  Let's say that $g$ in $f$ is {\i invocation-only } if $f$ not store $g$ in memory outside of $f$ scope and not return $g$ as part of return value. Only one action which $f$ can do with $g$ is invocation or passing as parameter to other invocation-only function.  If we look at Scala collection API, we will see, that near all hight-order functions there are invocation-only.

  Now, if we have $g$ which is invocation-only in $f$, and if we have function $g' : (A \To Future[B])$ let build function $f':(A\To Future[B])\to Future[C]$ that if $await(g')==await(g)$ then $await(f'(g'))==f(g))$ in next way
  \begin{itemize}
    \item $f'$ translated to $await(\makebox{transformed-body}(f))$
    \item $g(x)$ inside $f$ translated to $await(g'(x))$
    \item $h(g)$ translated to $h'(g')$ if $g$ is invocation-only in $h$.
  \end{itemize}

  Scala-gopher contains library of asynchronious variants of predefined functions, so it is possible to use asynchronious expressions inside loop. For example, next code:

\begin{Verbatim}[fontsize=\small]
  go { 
      for(i <- 1 to n) yield out.read
  }
\end{Verbatim}

  is transformed to

\begin{Verbatim}[fontsize=\small]
 async{ await {
   (1 to n).mapAsync(i => async{ await{ out.aread } } )
 } }
\end{Verbatim}

  which after simplification step become

\begin{Verbatim}[fontsize=\small]
   (1 to n).mapAsync(i => out.aread)
\end{Verbatim}

 Using this approach allows to overcome the inconvenience of async/await by allowing programmers use well-known API inside asynchronious expression. Also it is theoretically possible to generate asynchronious variants of functions by transforming TASTY representation of AST of synchronious versions. Simular technique is implemented in Nim \cite{Nim} programming language, where we can from one function definition generate both synchronious and asynchronious variants. 
  
 \item do transformation of defer statement.

\end{itemize}

\subsection{Channels: callbacks organized as waits}

  Channels in CSP are as two-sided pipes between processes; Channels messages not only pass information between goroutines but also coordinate process execution.
  In scala-gopher appropiative entities (\verb|Input[A]| for reading and \verb|Output[A]| for writing) implemented in fully asynchniously manner with help of callback-based interfaces:

\begin{Verbatim}[fontsize=\small]
trait Input[A]
{

   def  cbread[B](f:
            ContRead[A,B]=>Option[
                    ContRead.In[A]=>Future[Continuated[B]]
            ],
            ft: FlowTermination[B]): Unit
   ....
}
\end{Verbatim}

  Here we can read argument type as protocol where each arrow is a step : 
    $f$ is called on opportunity to read and
      $ContRead[A,B] \To Option[ContRead.In[A] \To Future[B]]$ means that when reading is 
   possible, we can ignore this opportunity (i.e. return None) or return handler which will
   consume value (or \verb|end-of-input| or few other special cases) and return future to the next 
   computation state.

  Traditional synchronious API  (i.e. method $read:\To A$) can be used inside \verb|go| and \verb|async| statements; from 'normal' code we can use asynchronious variant: $aread: \To Future[A]$.

  Output interaface is simular:
\begin{Verbatim}[fontsize=\small]
trait Output[A] 
{

  def  cbwrite[B](f: ContWrite[A,B] => Option[
                   (A,Future[Continuated[B]])
                  ],
                  ft: FlowTermination[B]): Unit
  
}
\end{Verbatim}
  Here $f$ is called on opportunity to write and when we decide to use this opportunity, we
 must provide actual value to write and next step in the same way as with \verb|Input|.

  Inputs and outputs are composable as can be expected in functional language and equipped by 
usual set of stream combinators: filter, map, zip, fold, .. etc.
 
  Channel is a combination of input and output. In addition to well-known buffered and unbuffered 
kinds of channels, scala-gopher provide some extra set of channels with different behaviour and performance characteristics,
 such as channel with growing buffer (a-la actor mailbox) or one-time channel based on \verb|Promise|, which is automatically closed after sending one message.

\subsection{Selectors: process composition as event generation}

 Mutualy exclusive process composition (i.e. determenistic choice: $(a \to P)\square(b\to Q)$ in original Hoar notation )  usualy represented in CSP=based languages as \verb|select| statement, which look's like ordinary switch. If you look in typical go program, you will discover often repeated code pattern: select inside endless loop inside go statement.

\begin{Verbatim}[fontsize=\small]
go {
  for{
    select{
     case c1 -> x :
           ......... // P
     case c2 <- y :
           ........  // Q
    }
  }
}
\end{Verbatim}

  Appropriative expression in CSP syntax: $*[(c_{1} ? x \to P)\square(c_{2} ! y \to Q)]$

 Scala-gopher provide \verb|select| pseudo-object which provide set of high-order pseudo-functions over
channels, which accept syntax of partial function over channel events:

\begin{Verbatim}[fontsize=\small]
go {   
  select.forever {
    case x: c1.read => ....  //P
    case y: c2.write => ....  //Q
  }
}
\end{Verbatim}
   
  or version which must not be wrapped by \verb|go| stamenet:

\begin{Verbatim}[fontsize=\small]
  select.aforever {
    case x: c1.read => ....  //P
    case y: c2.write => ....  //Q
  }
\end{Verbatim}
   
  Under the hood each such pseudo-function is build arround a flow (sequence of \verb|Continuated[_]| which represents step of computations optionally bound to channel event).  In unsugared form selector 

\begin{Verbatim}[fontsize=\small]
  val selector = SelectorForever()
  selector.onRead(ch)((x,ft,ec) => ... ) // P after go-transform
  selector.onWrite(ch,y)((y,ft,ec) => ... ) // Q after go-transform
  selector.run()
\end{Verbatim}

 We can maintain state inside a flow in a clear functional manner  using 'fold' family of select functions:

\begin{Verbatim}[fontsize=\small]
  def fibonacci(c: Output[Long], quit: Input[Boolean]): Future[(Long,Long)] =
     select.afold((0L,1L)) { case ((x,y),s) =>
      s match {
        case x: c.write => (y, x+y)
        case q: quit.read =>
                   select.exit((x,y))
      }
     }
\end{Verbatim}

  Here we see special syntax for tuple state. Also note, that \verb|afold| macro assume that \verb|s match| must be the first statement of pseudofunction. \verb|select.exit| is used for returning result from the flow.
 
  Events which we can check in select match statement are reading and writing of channels and select timeouts. In future we will think about extending set of notifications - i.e. adding channel closing and overflow notifications, which are rare needed in some scenarious.

\subsection{Transputer: entity which encapsulate processing node. }

 The idea is to have a actor-like object, which encapsulate processing node: ie read input data 
from set of input ports; write output to the set of output ports and maintaine a local 
mutable state inside.

Example:

\begin{Verbatim}[fontsize=\small]
class Zipper[T] extends SelectTransputer
{
 
   val inX: InPort[T]
   val inY: InPort[T]

   val out: OutPort[(T,T)] 

   loop {
     case x: inX.read => 
             val y = inY.read
             out write (x,y)
     case y: inY.read =>
             val x = inX.read
             out.write((x,y)) 
   }


}
\end{Verbatim}

  Having set of such objects, whe can build complex systems as combination of simple ones:
  \begin{itemize}
    \item $a+b$ - parallel execution; 
    \item $replicate[A](n)$ - transputer replication, where we start in parallel $n$ instances
  of $A$. Policy for prot replication can be configured - from sharing appropriative channel by each port to distributing   or duplication of signals to ditinguish each instance.
\begin{Verbatim}[fontsize=\small]
 val r = gopherApi.replicate[SMTTransputer](10)
    ( r.dataInput.distribute( (_.hashCode % 10 ) ).
       .controlInput.duplicate().
        out.share()
    )
\end{Verbatim}
    - here in r we have 10 instances of \verb|SMTTransputer|; if we send message to dataInput it will be send from one of instances in dependency from hashcode result, if we will send message to control input, it will be delivered to each instance; output channel will be shared between all instances.
  \end{itemize}

  Transputers can participate in error handling scenarious in the same way as actors: for each transputer we can define recovery policy and superviser.

\subsection{ Programming Techniques based on dynamic channels  }

 Let's outline some programming techniques, well known in Go world but not easy expressible in currwent mainstream scala streaming libraries. 

\begin{itemize}
 \item Channel expression as element of runtime state. Following this pattern allows developer to maintaine dynamics and potentialy recursive dataflows. 
 
 Example: 
  Let we have situation, where we need distribute some tasks accross set of relative slow consumers and we need to spawn additional consumers on peak usage and free resourses during calm. 

\begin{Verbatim}[fontsize=\small]
 select.fold(output){ (out, s) => s match {
   case x:input.read =>
     select.once {
       case x:out.write =>
       case select.timeout =>
             control.distributeBandwidth match {
                case Some(newOut) => newOut.write(x)
                                     (out | newOut)
                case None => control.report("Can't increase bandwidth")
                                     out
             }
     }
   case select.timeout =>
     out match {
       case OrOutput(frs,snd) => snd.close
                                 frs
       case _                 => out
     }
 } }
\end{Verbatim}

 Here we can request additional channels from control and construct merged channel in state. On read timeout we can deconstruct merged channel back and free unused resources.

\item{ Channel-based API where client supply channel where to pass reply }

Let we want to provide API which must on requiest return to client some value. Instead of providing method which will return result on the stack we can provide endpoint channel, which will accept method arguments and channel where to return result. 

 Next example illustrate this idea:

\begin{Verbatim}[fontsize=\small]

class Broadcast[T]
{

   val newListener: Channel[Channel[T]]
   val newMessage: Channel[T] = makeChannel[]

   def send(v:T):Unit = { newMessage.write(v)  }

   // private part
   case class Message(next:Channel[Message],value:T)

   select.afold(makeChannel[Message]) { (bus, s) =>
      s match {
         case v: newMessage.read =>
                   val newBus = makeChannel[Message]
                   current.write(Message(newBus,v))
                   newBus
         case ch: newListener.read =>          
                   select.afold(bus) { (current,s) =>
                     s match {
                       case msg:current.read =>
                               ch.awrite(msg.value) 
                               msg.next
                     }
                   } 
                   current
     } 
  }

}
\end{Verbatim}

 Broadcast provide API for creation of listeners and sending messages to all listeners as channel.
To register listener channel for receiving notification client sends this channel to newListener
  
 Internal state contains message bus represented by channel which is replaced during each new message and each listener spawns the process which reads message from current message bus.


\end{itemize}

 
\section{ Connection with other models }

 Exists many stream libraries for scala with different sets of tradeoffs. At one side of
spectrum we have clear streaming models like akka-streams\cite{akka-streams} with rich set
of composable operations and clear hight-level functionality but lack of flexibility,
from other side - very flexible but low-level models like actors.

 Scala-gopher provide uniform API which allows build systems from different parts of spectrum:
it is possible to build dataflow graph in declarative manner and connect one with dynamic part.

 The reactive isolates model\cite{Prokopec:2015:ICE:2814228.2814245} is close to scala-gopher model with dynamically-grown channel buffer (except that isolates supports distributed case).
 Isolate are corresponds to Transputers, Channel to Output part of gopher-channels and Events to goher Inputs. Channels in reactive isolates are more limited: only one isolate which own the channel can write to it, when in scala-gopher concept of channel ownity is absent. 

 Communicating Scala Objects\cite{CSO} is a direct implementation of CSP model in scala which allows
build expressions in internal scala DSL, closed to origin Hoar notation with some extensions, like extended renderzvous for mapping input streams.  Processes in CSO are not lightweight: each process require java thread which limit scalability of this library until lightweight threading will be implemented on JVM level.

 Subscript\cite{vanDelft:2013:DCL:2489837.2489849} is a scala extension which adds to language new constructions for building process algebra expressions. Althought extending language can afford fine-grained interconnection of process-algebra and imperative language notation in far perspective, now it's make CPA constructs second-class citizents because we have no direct representation of process and event types in scala type system.

  
\section{ Conclusion and future direction }
  
  Scala-gopher is a relative new library which yet not reach 1.0 state, but we have the early experience
 reports from using the library for building some helper components in industrial software project. 

 In general, feedback is positive: developers enjoy relative simple mental model and ability freely use asynchronious operations inside hight-order functions.  So, we can recommend to made conversion of invocation-only functions into async form to be generally available into async library itself. 

 
 
  TODO:  Go /  Distributed case

\section{Literature}

\printbibliography

\end{document}
